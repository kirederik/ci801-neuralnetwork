\documentclass[12pt]{article}

\usepackage{sbc-template}

\usepackage{graphicx,url}

\usepackage[brazil]{babel}   
\usepackage[utf8x]{inputenc}  

     
\sloppy

\title{Evoluindo os pesos de uma Rede Neural \\com Algoritmos Genéticos}

\author{Aurora Trinidad R. Pozo\inst{1}, Davi Azevedo Q. Santos\inst{1}, Derik Evangelista R. Silva\inst{1}}
  

\address{Departamento de Informática -- Universidade Federal do Paraná (UFPR)\\
  Caixa Postal 19081  -- 81531-980 -- Curitiba -- PR -- Brasil
  \email{\{aurora, daqsantos, dersilva\}@inf.ufpr.br}
}

\begin{document} 

\maketitle

\begin{abstract}
  This meta-paper describes the style to be used in articles and short papers
  for SBC conferences. For papers in English, you should add just an abstract
  while for the papers in Portuguese, we also ask for an abstract in
  Portuguese (``resumo''). In both cases, abstracts should not have more than
  10 lines and must be in the first page of the paper.
\end{abstract}
     
\begin{resumo} 
  Este meta-artigo descreve o estilo a ser usado na confeco de artigos e
  resumos de artigos para publicao nos anais das conferncias organizadas
  pela SBC.  solicitada a escrita de resumo e abstract apenas para os artigos
  escritos em portugus. Artigos em ingls devero apresentar apenas abstract.
  Nos dois casos, o autor deve tomar cuidado para que o resumo (e o abstract)
  no ultrapassem 10 linhas cada, sendo que ambos devem estar na primeira
  pgina do artigo.
\end{resumo}


\section{Introdução}

Contextualização do trabalho: 
\begin{itemize}
\item Contexto histórico de IA/Aprendizado de máquina
\item Motivação
\item Objetivos
\item Linha do trabalho
\end{itemize}

\section{Metodologia} \label{sec:metodologia}

\begin{itemize}
\item Desenvolvimento do trabalho
\item Linguagem utilizada
\item Tratamento das bases
\item etc
\end{itemize}

\section{Redes Neurais}

\begin{itemize}
\item Histórico
\item Breve explicação de conceitos relevantes ao trabalho
\end{itemize}

\section{Algoritmos Genéticos}

\par O Algoritmo Genético (AG), geralmente referenciado como \emph{algoritmos genéticos}, foi desenvolvido por John Holland na Universidade de Michigan, em 1975, em um dos livros mais famosos da área, \emph{Adaptation in Natural and Artificial Systems}, publicado pela editora \emph{University of Michigan Press} \cite{essentials:pop}.
\par Os AGs são algoritmos de busca heurística que tentam reproduzir artificialmente o processo de evolução, da Teoria de Evolução das Espécies, de Darwin. São muito similares a outros algoritmos de busca local, como o Subida de Encosta (\textit{Hill-climbing}), diferindo apenas na quantidade de soluções mantidas a cada iteração e no processo de geração de novas soluções.
\par De acordo com \cite{Montana}, os AGs necessitam de cinco componentes:
\begin{enumerate}
\item[C1] - Uma maneira de codificar uma solução em um indivíduo.
\item[C2] - Uma função de avaliação que retorna um índice de qualidade para cada individuo da população.
\item[C3] - Um procedimento de inicialização da população.
\item[C4] - Operadores que podem ser aplicados nos indivíduos pais no processo de reprodução, alterado a composição genética dos novos indivíduos gerados. Neste componente incluem-se os operadores de mutação, de cruzamento e outros específicos do domínio.
\item[C5] - Uma configuração de parâmetros do algoritmo, os operadores, etc.
\end{enumerate}
Com estes cinco componentes, ainda de acordo com \cite{montana}, o AG funciona seguindo os seguintes passos:
\begin{enumerate}
	\item A população é iniciada usando-se o procedimento C3. O resultado é um conjunto de indivíduos, de acordo com C1.
	\item Cada indivíduo é avaliado, usando a função definida em C2.
	\item A população se reproduz até que um critério de parada seja atingido. A reprodução é realizada seguindo-se os seguintes passos:
	 \begin{enumerate}
		\item Um ou mais indivíduis são escolhidos para a reprodução. A seleção é estocástica, mas os pais melhores avaliados são favorecidos na escolha. Os parametros em C5 podem influenciar neste processo de seleção.
		\item Os operadores de C4 são aplicados aos pais para geração dos filhos. Os parametros em C5 ajudam a determinar quais operadores serão usados.
		\item Os filhos são então avaliados e inseridos de volta na população. Em algumas versões de AG, toda a população é substituida. Em outras, apenas um subconjunto é substituido.		
	\end{enumerate}
\end{enumerate}

\par A cada iteração, os indivíduos melhor avaliados tem mais chances de serem escolhidos para reprodução, fazendo com que, em teoria, sejam gerados melhores indivíduos a cada geração. Ao término do algoritmo, senão a ótima, a tendência é ter-se uma solução muito próxima a ela.

\section{Implementação}\label{sec:imple}

\begin{itemize}
\item Detalhes da nossa implementação
\end{itemize}

\section{Resultados obtidos}

Resultados obtidos e comparação com os resultados dos artigos estudados

\section{Considerações Finais}

Conclusões que refletem o objetivo do trabalho


\bibliographystyle{sbc}
\bibliography{sbc-template}

\end{document}
