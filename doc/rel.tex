\documentclass[12pt]{article}

\usepackage{sbc-template}

\usepackage{graphicx,url}

\usepackage[brazil]{babel}   
\usepackage[utf8x]{inputenc}  

     
\sloppy

\title{Evoluindo os pesos de uma Rede Neural \\com Algoritmos Genéticos}

\author{Aurora Trinidad R. Pozo\inst{1}, Davi Azevedo Q. Santos\inst{1}, Derik Evangelista R. Silva\inst{1}}
  

\address{Departamento de Informática -- Universidade Federal do Paraná (UFPR)\\
  Caixa Postal 19081  -- 81531-980 -- Curitiba -- PR -- Brasil
  \email{\{aurora, daqsantos, dersilva\}@inf.ufpr.br}
}

\begin{document} 

\maketitle

\begin{abstract}
  This meta-paper describes the style to be used in articles and short papers
  for SBC conferences. For papers in English, you should add just an abstract
  while for the papers in Portuguese, we also ask for an abstract in
  Portuguese (``resumo''). In both cases, abstracts should not have more than
  10 lines and must be in the first page of the paper.
\end{abstract}
     
\begin{resumo} 
  Este meta-artigo descreve o estilo a ser usado na confeco de artigos e
  resumos de artigos para publicao nos anais das conferncias organizadas
  pela SBC.  solicitada a escrita de resumo e abstract apenas para os artigos
  escritos em portugus. Artigos em ingls devero apresentar apenas abstract.
  Nos dois casos, o autor deve tomar cuidado para que o resumo (e o abstract)
  no ultrapassem 10 linhas cada, sendo que ambos devem estar na primeira
  pgina do artigo.
\end{resumo}


\section{Introdução}

Contextualização do trabalho: 
\begin{itemize}
\item Contexto histórico de IA/Aprendizado de máquina
\item Motivação
\item Objetivos
\item Linha do trabalho
\end{itemize}

\section{Metodologia} \label{sec:metodologia}

\begin{itemize}
\item Desenvolvimento do trabalho
\item Linguagem utilizada
\item Tratamento das bases
\item etc
\end{itemize}

\section{Redes Neurais}

\begin{itemize}
\item Histórico
\item Breve explicação de conceitos relevantes ao trabalho
\end{itemize}

\section{Algoritmos Genéticos}

\begin{itemize}
\item Histórico
\item Breve explicação de conceitos relevantes ao trabalho
\end{itemize}

\section{Implementação}\label{sec:imple}

\begin{itemize}
\item Detalhes da nossa implementação
\end{itemize}

\section{Resultados obtidos}

Resultados obtidos e comparação com os resultados dos artigos estudados

\section{Considerações Finais}

Conclusões que refletem o objetivo do trabalho


\bibliographystyle{sbc}
\bibliography{sbc-template}

\end{document}
